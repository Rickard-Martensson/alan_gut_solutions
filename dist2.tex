\documentclass{article}
\usepackage[utf8]{inputenc}
\usepackage{amsmath, amssymb}
\usepackage{booktabs}
\usepackage{pdflscape}
\usepackage{geometry}
\usepackage{mathtools}
\DeclarePairedDelimiter\ceil{\lceil}{\rceil}
\DeclarePairedDelimiter\floor{\lfloor}{\rfloor}

\geometry{left=1cm,right=1cm,top=1.5cm,bottom=1.5cm} % Adjust margins as needed

\begin{document}

\begin{landscape}
    \begin{table}[ht]
        \centering
        \caption{Summary of Discrete Distributions}
        \label{tab:probability_distributions}
        
        \begin{tabular}{@{}l p{6.5cm} p{4.5cm} p{3.0cm} r@{}}
        \toprule
        Distribution, notation & Probability Function & Cumulative Distribution & Statistics & Generating Functions \\ 
        \midrule

    
        $\begin{array}{l}
            \textbf{One point} \\
            \delta(a) \\
    \end{array}$ & 
    $\begin{array}{l}
    \displaystyle p(a) = 1\\
    \end{array}$ & 
    $\begin{array}{l}
        \displaystyle F_X(k) = 0, \text{if } a < 0\\
        \displaystyle F_X(k) = q, \text{if } a \geq 0 \\
    \end{array}$ & 
    $\begin{array}{l}
    \displaystyle \hspace{0.36cm}  EX = a \\
    \displaystyle VarX = 0
    \end{array}$ & 
    $\begin{array}{r}
    \displaystyle g_X(t) = ta \\
    \displaystyle \psi_X(t) = e^{ta} \\
    \displaystyle \varphi_X(t) = e^{ita}
    \end{array}$ \\
    \multicolumn{4}{p{19cm}}{Pretentious to have this, i know, but i guess it needs to be complete or whatevs} \\


        $\begin{array}{l}
            \textbf{Bernoulli} \\
            B(p) \\
            0 \leq p \leq 1
    \end{array}$ & 
    $\begin{array}{l}
    \displaystyle p(0) = q, p(1) = p; q = 1-p\\
    \end{array}$ & 
    $\begin{array}{l}
        \displaystyle F_X(k) = 0, \text{if } k < 0\\
        \displaystyle F_X(k) = q, \text{if } 0\leq k < 1 \\
        \displaystyle F_X(k) = 1, \text{if } k \geq 1 \\
    \end{array}$ & 
    $\begin{array}{l}
    \displaystyle \hspace{0.36cm}  EX = p \\
    \displaystyle VarX = pq
    \end{array}$ & 
    $\begin{array}{r}
    \displaystyle g_X(t) = q+pt \\
    \displaystyle \psi_X(t) = q+pe^{t} \\
    \displaystyle \varphi_X(t) = q+pe^{it}
    \end{array}$ \\
    \multicolumn{4}{p{19cm}}{Lorem ipsum} \\
    

    $\begin{array}{l}
        \textbf{Binomial} \\
        B(n,p) \\
        n = 1,2,...;0 \leq p \leq 1
\end{array}$ & 
$\begin{array}{l}
\displaystyle p(k) = \binom{n}{k} p^kq^{n-k} ; k = 0,1,...n; q = 1-p
\end{array}$ & 
$\begin{array}{l}
    {\displaystyle I_{q}(n-\lfloor k\rfloor ,1+\lfloor k\rfloor )} 
\end{array}$ & 
$\begin{array}{l}
\displaystyle \hspace{0.36cm}  EX = np \\
\displaystyle VarX =  npq
\end{array}$ & 
$\begin{array}{r}
\displaystyle g_X(t) = (q+pt)^n \\
\displaystyle \psi_X(t) = (q+pe^{t})^n \\
\displaystyle \varphi_X(t) = (q+pe^{it})^n
\end{array}$ \\
\multicolumn{4}{p{19cm}}{Lorem ipsum} \\

    
$\begin{array}{l}
    \textbf{Poisson} \\
    Po(\lambda) \\
    \lambda > 0
\end{array}$ & 
$\begin{array}{l}
\displaystyle p(k) = \frac{\lambda^k}{k!} e^{-\lambda}, k = 0,1,2...\\
\end{array}$ & 
$\begin{array}{l}
\displaystyle F_X(x) = 1 - q^{\floor{x} + 1}
\end{array}$ & 
$\begin{array}{l}
\displaystyle \hspace{0.36cm}  EX = \lambda \\
\displaystyle VarX = \lambda
\end{array}$ & 
$\begin{array}{r}
\displaystyle g_X(t) = exp\{t-1\} \\
\displaystyle \psi_X(t) =  exp\{ e^t-1\} \\
\displaystyle \varphi_X(t) = exp\{ e^{it}-1\}
\end{array}$ \\
\multicolumn{4}{p{19cm}}{Lorem ipsum} \\



        $\begin{array}{l}
                \textbf{Geometric} \\
                G(p) \\
                0 \leq p \leq 1
        \end{array}$ & 
        $\begin{array}{l}
        \displaystyle p(k) = p q^{k}, k = 0,1,2... ; q = 1-p\\
        \end{array}$ & 
        $\begin{array}{l}
            \displaystyle F_X(x) = 1 - q^{\floor{x} + 1}
        \end{array}$ & 
        $\begin{array}{l}
        \displaystyle \hspace{0.36cm}  EX = \frac{q}{p} \\
        \displaystyle VarX = \frac{q}{p^2}
        \end{array}$ & 
        $\begin{array}{r}
        \displaystyle g_X(t) = \frac{p}{1-qt} \\
        \displaystyle \psi_X(t) = \frac{p}{1-qe^{t}} \\
        \displaystyle \varphi_X(t) = \frac{p}{1-qe^{it}}
        \end{array}$ \\
        \multicolumn{4}{p{19cm}}{Lorem ipsum} \\
        
        $\begin{array}{l}
            \textbf{First success} \\
            G(p) \\
            0 \leq p \leq 1
    \end{array}$ & 
    $\begin{array}{l}
    \displaystyle p(k) = p q^{k-1}, k = 1,2... ; q = 1-p\\
    \end{array}$ & 
    $\begin{array}{l}
        \displaystyle F_X(x) = 1 - q^{\floor{x}}
    \end{array}$ & 
    $\begin{array}{l}
    \displaystyle \hspace{0.36cm}  EX = \frac{q}{p} \\
    \displaystyle VarX = \frac{q}{p^2}
    \end{array}$ & 
    $\begin{array}{r}
    \displaystyle g_X(t) = \frac{pt}{1-qt} \\
    \displaystyle \psi_X(t) = \frac{pe^{t}}{1-qe^{t}} \\
    \displaystyle \varphi_X(t) = \frac{pe^{it}}{1-qe^{it}}
    \end{array}$ \\
    \multicolumn{4}{p{19cm}}{Identical to Geometric, but always 1 larger} \\
        
        \bottomrule
        \end{tabular}
        \end{table}


\begin{table}[ht]
\centering
\caption{Summary of Probability Distributions}
\label{tab:probability_distributions}

\begin{tabular}{@{}l p{6.0cm} p{5cm} p{3.0cm} r@{}}
\toprule
Distribution, notation & Density Function $f_X(x)$ & Cumulative Distribution $F_X(x)$ & Statistics & Generating Functions \\ 
\midrule
$\begin{array}{l}
        \textbf{Normal} \\
        \mathcal{N}(\mu, \sigma^2) \\
        \sigma > 0
\end{array}$ & 
$\begin{array}{l}
\displaystyle  \frac{1}{\sqrt{2\pi \sigma^2}} exp \Bigl\{  -\frac{1}{2} \frac{(x-\mu)^2}{\sigma^2}\Bigr\}\\
\end{array}$ & 
$\begin{array}{l}
    \displaystyle  \Phi \left(\frac{x-\mu}{\sigma}\right)
\end{array}$ & 
$\begin{array}{l}
\displaystyle \hspace{0.36cm}  EX = \mu \\
\displaystyle VarX = \sigma^2
\end{array}$ & 
$\begin{array}{r}
\displaystyle \psi_X(t) = e^{t\mu + \frac{1}{2} \sigma^2 t^2} \\
\displaystyle \varphi_X(t) = e^{it\mu - \frac{1}{2} \sigma^2 t^2}
\end{array}$ \\
\multicolumn{4}{p{19cm}}{The Normal distribution is central to the central limit theorem} \\


$\begin{array}{l}
    \textbf{Exponential} \\
    Exp(\lambda) \\
    \lambda > 0
\end{array}$ & 

$\begin{array}{l}
\displaystyle   \frac{1}{\lambda} e^{-x / \lambda} \\
\end{array}$ & 
$\begin{array}{l}
    \displaystyle  1-e^{-x / \lambda}
\end{array}$ & 
$\begin{array}{l}
\displaystyle \hspace{0.36cm} EX = \lambda \\
\displaystyle VarX = \lambda^2
\end{array}$ & 
$\begin{array}{r}
\displaystyle \psi_X(t) = \frac{1}{(1 - \lambda t)} \\
\displaystyle \varphi_X(t) = \frac{1}{(1 - \lambda it)}
\end{array}$ \\
\multicolumn{4}{p{19cm}}{Gamma is crucial in modeling waiting times and life data.} \\



$\begin{array}{l}
    \textbf{Gamma} \\
    \Gamma(p, a) \\
    a > 0, p > 0
\end{array}$ & 

$\begin{array}{l}
\displaystyle  \frac{1}{\Gamma(p)} x^{p-1} \frac{1}{a^p} e^{-x/a} \\
\end{array}$ & 
$\begin{array}{l}
    \displaystyle  \frac{1}{\Gamma(p)} \gamma \left(p, \frac{x}{a}\right)
\end{array}$ & 
$\begin{array}{l}
\displaystyle \hspace{0.36cm} EX = pa \\
\displaystyle VarX = pa^2
\end{array}$ & 
$\begin{array}{r}
\displaystyle \psi_X(t) = \frac{1}{(1 - a t)^{p}} \\
\displaystyle \varphi_X(t) = \frac{1}{(1 - a it)^{p}}
\end{array}$ \\
\multicolumn{4}{p{19cm}}{Gamma is crucial in modeling waiting times and life data.} \\

$\begin{array}{l}
    \textbf{Laplace} \\
    L(a) \\
    n = 1, 2,...
\end{array}$ & 

$\begin{array}{l}
\displaystyle \frac{1}{2a}e^{-|x|/a}, -\infty < x <\infty \\
\end{array}$ & 
$\begin{array}{l}
   {\displaystyle {\begin{cases}{\frac {1}{2}}\exp \left({\frac {x-\mu }{b}}\right)&{\text{if }}x\leq \mu \\[8pt]1-{\frac {1}{2}}\exp \left(-{\frac {x-\mu }{b}}\right)&{\text{if }}x\geq \mu \end{cases}}}
\end{array}$ & 
$\begin{array}{l}
\displaystyle \hspace{0.36cm} EX = 0 \\
\displaystyle VarX = 2a^2\\
\end{array}$ & 
$\begin{array}{r}
\displaystyle \psi_X(t) = {\displaystyle {\frac {1}{1-a^{2}t^{2}}}}, |t|<1/a\\
\displaystyle \varphi_X(t) = {\displaystyle {\frac {1}{1+a^{2}t^{2}}}}
\end{array}$ \\
\multicolumn{4}{p{19cm}}{Laplace is kinda cool} \\


$\begin{array}{l}
    \textbf{Beta} \\
    \beta(r,s) \\
    r,s > 0
\end{array}$ & 

$\begin{array}{l}
    {\displaystyle {\frac {\Gamma (r+s)}{\Gamma (r) \Gamma (s)}x^{r-1} (1-x)^{s-1}}, 0<x<1}
\end{array}$ & 
$\begin{array}{l}
   {\displaystyle No. }
\end{array}$ & 
$\begin{array}{l}
\displaystyle \hspace{0.36cm} EX =\frac{r}{r+s} \\
\displaystyle VarX = \frac{rs}{(r+s)^2(r+s+1)}\\
\end{array}$ & 
$\begin{array}{r}
\displaystyle \psi_X(t) = {\displaystyle 1+\sum _{k=1}^{\infty }\left(\prod _{r=0}^{k-1}{\frac {\alpha +r}{\alpha +\beta +r}}\right){\frac {t^{k}}{k!}}}\\
\displaystyle \varphi_X(t) = {\displaystyle {\frac {1}{1+a^{2}t^{2}}}}
\end{array}$ \\
\multicolumn{4}{p{19cm}}{Laplace is kinda cool} \\




$\begin{array}{l}
    \textbf{(Student's) t} \\
    t(n) \\
    n = 1, 2,...
\end{array}$ & 

$\begin{array}{l}
\displaystyle \frac{\Gamma\left(\frac{n+1}{2}\right)}{\sqrt{\pi n} \Gamma\left(\frac{n}{2}\right)} \left(1+\frac{x^2}{n}\right)^{-\frac{n+1}{2}} \\
\end{array}$ & 
$\begin{array}{l}
    \displaystyle No. \\
\end{array}$ & 
$\begin{array}{l}
\displaystyle \hspace{0.36cm} EX = 0 \\
\displaystyle VarX = \frac{n}{n-2}, n>2\\
\end{array}$ & 
$\begin{array}{r}
\displaystyle \psi_X(t) = undefined \\
\displaystyle \varphi_X(t) = No.
\end{array}$ \\
\multicolumn{4}{p{19cm}}{Student t is crucial in modeling waiting times and life data. The inventor worked at Guiness, and published under a pseudonym so that their competitors wouldn't know they were using advanced statsistics} \\


\bottomrule
\end{tabular}
\end{table}
\end{landscape}

\end{document}
